\section{Related Work}

In this section we discuss some of the existing research in the area of ranking in social structure, with respect to minorities. Minority ranking in networks has been studied by many researchers owing to the fact that these rankings help understand a minority group effect in a network. In \cite{karimi2018homophily}, Karimi et. al. shows how different rates of homophilic behavior and group sizes affects degree ranking of nodes in a network. The minority groups are at a disadvantage in case of an extremely homophilic network. Perception biases have been studied with respect to group size and homophily in \cite{lee2017homophily}. It has also been seen in \cite{stoica2018algorithmic} where recommendations are studied to see how the algorithmic growth of a network brings out the glass ceiling effect for minorities. 

Learning-To-Rank with respect to Reinforcement Learning has been studied for document retrieval, search-query rankings. These algorithms learn from click-behavior models to rank documents for queries. Click-models are stochastic models which approximate from click logs user interaction with list of items. The Multi-Armed Bandits sub-field of reinforcement learning is used to describe this problem, where there is a single state and given certain actions (or arms), which action selection would lead to the best outcome. Some of the state-of-the-art algorithms to learn ranking of online documents are - BatchRank \cite{zoghi2017online} by Zoghi et al., TopRank \cite{lattimore2018toprank} by Lattimore et. al, BubbleRank \cite{li2019bubblerank} by Li et al. BubbleRank is currently superior to TopRank, which is shown to be superior to BatchRank, which itself is superior to RankedExp3 \cite{radlinski2008learning} by Radlinski et al. which was one of the first papers to start exploring the problem of ranking documents for a query with reinforcement learning. BubbleRank however only helps in the re-ranking process, along with helping in other problems like the warm-start problem. 

The existing literature focuses on document retrieval for queries, and we do not find any research which focuses on link prediction or recommendation of social ties in a online social network setting. Thus we derive our work mostly from these document retrieval algorithms, trying to fit them in our social links recommendation scenario. 