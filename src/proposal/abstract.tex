\begin{abstract}
	Homophily is a common effect experienced in many real-world social networks today. Minority group ranking with the effect of homophily has been extensively studied by several researchers over the past years. In today's world most of the social networks connections are formed through recommendation systems. These systems learn global and specific patterns in user behavior to recommend potential network growth opportunities. In recent years, the field of Reinforcement Learning has become very popular for different use-cases, along with recommendation algorithms. In this work we try to analyze how the learning agent which learns to provide recommendations aiding the growth of network is affected by the homophilic behavior of users and how this in turn affects minority ranking in the network. This would help in better understanding network growth and group disparities by showing the effect of a reinforcement learning agent. Our analyses will compare and contrast with popular models used currently for network growth, and try to come up with ways if minority ranking can be improved from an algorithmic point of view. 
	
\keywords{Social Recommender \and Homophily \and Reinforcement learning}

\end{abstract}