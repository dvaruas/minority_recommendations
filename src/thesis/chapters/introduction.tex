\chapter{Introduction}
\label{introduction}
\thispagestyle{empty}

It is a well known fact that human beings are social animals. Forming communities leads to a sense of belonging and trust which has led humans to thrive in our evolutionary growth overcoming challenging adversities. This idea of belongingness is so inherent in our nature that it has been identified as one of the basic psychological needs by Maslow in the human hierarchy of needs \cite{maslow1958dynamic}. With the advent of internet and its widespread use in our daily lives this aspect was quick to find new grounds in the virtual world, somewhere around early 2000's. These are Online Social networks, platforms which help users get connected with people across the world by allowing them to maintain user profiles and upload generated content to the platform which can then be shared across the user base. While different online platforms cater to varied needs, the underlying motivation is the same - \textit{``to connect, to belong''}. While there are platforms like Facebook\footnote{www.facebook.com}, Twitter\footnote{www.twitter.com} or Instagram\footnote{www.instagram.com} which are used primarily in an informal setting, there are also platforms like LinkedIN\footnote{www.linkedin.com} which are used in a more professional one. So it is evident that these networks impact humans and their lives in different fronts.

Nearly every social network today is coupled with a recommender system. It has become the dominant mechanism through which an individual can get connected to new people, content, products or events; can expand his/her own knowledge-base, or simply find new avenues for exploration within the given social space. As internet has grown the amount of information flowing through the space has become enormous, and finding relevant information has come out as a big challenge. Even with the information being organized based on content, relevance ranking based on user preference is challenging - a task which has been taken up by these recommendation systems \cite{rashid2002getting}. Undoubtedly this has enhanced the user experience in the platforms considerably and user engagement has been observed to be much higher with the introduction of such agents \cite{pu2011user, konstan2012recommender}.

Research in the field of recommendation systems had been widely popularized by the Netflix Prize competition \cite{netflixprize2006} back in 2006. Since then, much of the research in literature has focused on content recommendation since the business aspect of it is quiet evident. However, another popular variant of recommendation systems focus on recommending users. These agents have been found by researchers to play a very crucial role in the growth of social networks \cite{su2016effect,daly2010network,stoica2018algorithmic}. Some of these agents from some of the most popular social networking platforms like Facebook, Twitter, Instagram are ``People you may know'', ``Who to Follow'', ``Suggested people'' respectively. These recommender systems in social networks when suggesting links uses link prediction strategies derived from the network structure and/or node and edge attributes. It is unclear how most of the recommender systems work in these commercial social networks owing to trade secrets, but for some of them we do have partial information, like Twitter's Who-To-Follow agent \cite{gupta2013wtf} and Facebook's People-You-May-Know \cite{backstrom2011supervised}. These agents mainly exploit friends-of-friends strategies (there is a higher probability of people sharing common friends), or random walk models to predict links. These approaches have been quite common in research and has been extensively studied by various researchers in \cite{backstrom2011supervised,chen2009make,gupta2013wtf}. At the heart of these systems lies the triadic closure concept from social sciences theories which was first proposed by the German sociologist Georg Simmel \cite{simmel1950sociology}. The way these network grows under such recommendation algorithms has been analyzed by Su et. al. in \cite{su2016effect}, where they have observed the power-law construct with `rich-getting-richer' effect materializing in networks under the influence of such systems.

The crucial role recommender systems play in networks have been known to borne several ill effects: from creating highly polarized networks, recognizable glass ceiling effects, to trapping users in ideological echo chambers \cite{stoica2018algorithmic}. These ill-effects have also often been pointed out by multiple news sources \cite{youtubefeed2020,guardianselfharm,youtuberabbit2020} and demands have often been made by several public and government organizations for remands from these tech giants. The recommender systems are primarily built with the goal of maximizing user engagement with the platform. However, nowadays companies are trying to tweak their AI models to have better judgement and not be overtly biased to reaffirm biases from hyper-engaged users. Although the process has started, there's still a long way to go for industries to commit to developing technologies which do not conform to age-old biases and are responsible to a certain extent for the choices they make for us.

Also with respect to social networks, homophily as a parameter has been found to be quite relevant in shaping the evolution of network structures \cite{dong2017structural,mcpherson2001birds}. The term homophily was coined by Lazarsfeld et al. in their essay ``Friendship as a social process'' \cite{lazarsfeld1954friendship}. The essence of homophily can perhaps be best understood through the famous proverb -  \textit{``Birds of a feather flock together''}. This parameter tries to capture and  quantify the human instinct to bond with people who share common attributes with them. These attributes could be as varied as common race, age-group, gender, common goals, ideologies or personalities. Humans being communal by nature tend to form communities and these aforementioned attributes are a way of building associations. Homophily in social structures, specially how these associations affect a given minority and majority group has been widely researched. It has been well established through various studies that these have varying effects on the evolving structure of a society, distribution of information and visibility of a certain group \cite{stoica2018algorithmic,avin2015homophily,mcpherson2001birds,karimi2018homophily}. 

As machine learning systems learn from user behavior, it can be expected that the intrinsic homophilic patters in humans behavior becomes a part of the algorithmic behavior too. Networks growing with the help of these biased recommendation systems would therefore imbibe this bias into the network structure itself. When a supposedly objective system would need to take some decision based on the parameters of these kind of evolved networks, it will fail to maintain its objectivity due to the inherent bias a network might possess. Ranking systems can be quite prevalent when it comes to making decisions for `employability', `credit-score' or other services \cite{heap2014combining,kalayci2018credit}. These systems are supposed to objectively rank people based on certain parameters. These kind of systems would be affected by a biased model of the network. This forms the crux of our work, trying to understand how a recommender agent aided growth of a network is influenced by homophily of the users interacting with it. 

Along with established link prediction models, we use reinforcement learning methods in our study. Our usage of reinforcement learning methods is stemmed from the fact that this branch of learning algorithms is widely gaining popularity due to the promising results it shows. Although there are multiple challenges currently in having a reinforcement learning agent function effectively in the real-world (as has been studied in \cite{dulac2019challenges}) we can expect to see these kind of systems more in picture over the years - much like Alibaba using RL agent for the first time in their platform Taobao\footnote{https://world.taobao.com/}. The results from their research shows good performance in the online community over other popular ML approaches \cite{jin2018real,shi2019virtual} and thus shows huge promise in this area. Russell and Norvig in their book `Artificial Intelligence' comments \textit{``reinforcement Learning might be considered to encompass all of AI: an agent is placed in an environment and must learn to behave successfully therein and reinforcement learning can be viewed as a microcosm for the entire AI problem''} \cite{russell2016artificial}. Reinforcement learning is the very basis by which us humans, the most effective learning agents learn to tackle novel situations in our environment. Advances in Deep RL has paved the path further, invoking the popular hypothesis by David Silver that \textit{``AI = RL + DL''} \cite{davidsilvertut}. This makes a strong case to use reinforcement learning agents for recommendations and study their effect on network structure, specially with the combination of homophily.

In this thesis, we consider a total of 5 recommendation methods, which we use to study the bias it induces onto a network. We use both synthetic and empirical networks in our experiments, both in a \textit{static} and \textit{growing} setting. In a \textit{static} setting we observe the recommendation bias for a given network snapshot. For calculating the recommendation bias we use the Disparate Visibility measure by Fabbri et. al. \cite{fabbri2020effect} which observes visibility of a group in proportion to its relative size. In our \textit{growing} setting we observe the network growth with the aid of a recommendation system and see how the network is shaped over a period of time. Finally, we also use our recommendation agents to observe the disparate visibility on some empirical Facebook networks in this thesis.

\section{Research Goals}
\label{research_goals}
Having introduced the broader scope, we define the following research goals which we wish to explore in our thesis.

\begin{itemize}
	\item[\textbf{RG1.1}\label{rg1.1}] \textit{Perform bias analysis on recommendations for pre-generated synthetic network.}
	
	We use the generative model proposed by Karimi et. al. \cite{karimi2018homophily} to generate synthetic networks with different homophily and minority fraction parameters. These generated networks mimic observable properties of real-world online social networks and thus form a good base for our analysis. Our goal is to use different recommendation algorithms for every node in a given network and upon receiving these recommendations perform a bias analysis on them using the disparate visibility measure.
	
	\item[\textbf{RG1.2}\label{rg1.2}] \textit{Perform bias analysis on recommendations obtained for empirical networks.} 
	
	We choose 4 different empirical networks from the Facebook100 dataset \cite{traud2012social} to obtain recommendations for the network nodes using different recommendation algorithms. For the chosen networks we select gender as the defining attribute to classify the network nodes into a majority and a minority group. Our goal is similar to \textbf{RG1.1} in performing bias analysis on the obtained recommendations using the disparate visibility measure.
	
	\item[\textbf{RG2}\label{rg2}] \textit{Observe the structure and properties of a synthetic network grown using different recommendation methods.}
	
	We define a synthetic network generation algorithm inspired by the growing model defined by Stoica et. al. \cite{stoica2018algorithmic}. The growth of this network is defined by two distinct types of growth - organic and algorithmic. \textit{Organic growth} signifies the addition of new nodes to the network and choosing nodes to form edges with, without the influence of recommendation system. \textit{Algorithmic growth} is completely driven by the recommendation agent and helps in formation of new links among existing nodes in the network. The interplay of these two growth types help us mimic how real world networks are formed over time with the aid of recommendation agents. Our goal is to tune different parameters such as minority fraction, homophily and choice of the recommendation system (along with the degree to which a rank index is given preference) and observe the network growth. This would help us in understanding the evolution of networks under the influence of different recommendation systems. 
	
\end{itemize} 

\section{Contributions}
Through our thesis, we introduce the novel concept of using reinforcement learning methods (which have been primarily used for ranking search query results) as a potential new way of ranking nodes for recommendation. We define our own theoretical click model which is used by these methods during their training phase. Due to the lack of empirical data related to user recommendations, we are unable to accurately capture the clicking behavior existent in real-world. However this gives a proper base, and it would be quite easy for any commercial platform to gather such data and use it for their own recommendation engine. We observe that the networks formed with these kind of recommendation agents however reflect the inherent bias of the users in the network, and this hints towards the use of better click models to help networks grow in a more unbiased structure.

We have also performed comparative studies on the effect of different recommendation agents in structuring social networks over time by changing several parameters like homophily and minority fraction to generate synthetic networks. This gives us a deeper understanding of the effect these systems would be having, if used. For the reinforcement learning models it can be well said that their bias is completely dependent on the clicking model being used. For other popular recommendation methods, the working of the individual methods produces different results and we see that attribute agnostic models provide significantly better unbiased networks over time.

Finally, through our study of the empirical Facebook networks we get an idea of the disparity in visibility present in real-world network recommendations.

\section{Outline}
In chapter \ref{chapter_background} an overview of previously related work in the area is provided along with establishing the necessary background for our thesis. In chapter \ref{recommender_methods} we discuss in detail the various recommender systems we plan to use in our study. Following that, in chapter \ref{analysis_chapter}, we work with synthetic data for our research goals \textbf{RG1.1} and \textbf{RG2}. We formally define the experimental setup along with the algorithms used for the generation of our synthetic networks. An analysis of the results and our observations is then provided. In chapter \ref{case_study} we work on our research goal \textbf{RG1.2} by defining the datasets being used and documenting the various pre-processing steps and analysis of the bias found in the recommendations for them. In chapter \ref{discussions}, we discuss limitations of our study and provide directions for future work. Finally, we conclude our thesis in chapter \ref{conclusions} with a brief overview of our study.